\subsection{Pseudocódigo}

Del enunciado del problema se pueden deducir ciertos datos que nos van a ayudar a dar con el algoritmo que buscamos.

Aunque resulte un poco obvio a esta altura, la forma de interpretar el problema es por medio de un grafo, en principio no orientado, donde los nodos representan servidores y las aristas los enlaces entre sí.

Sabemos por el enunciado que la entrada que recibimos, es la salida del ejercicio anterior, por lo tanto podemos afirmar:

\begin{itemize}
\item Desde un servidor (cualquiera) se debe poder llegar a todos los demás.
\item La cantidad de enlaces usados es mínima.
\end{itemize}

De lo primero se deduce que el grafo es conexo, no importa que par de nodos elija, existe un camino que los comunica.

El segundo punto nos permite afirmar que no hay circuitos simples (camino que salga de un nodo y vuelva al mismo, pasando a lo sumo una vez por cada arista). Cada nodo dentro del circuito tendría al menos dos formas de conectarse con otro nodo cualquiera   entonces podría sacar una arista adyacente que pertenezca a uno de esos caminos y el grafo seguiría estando conexo.

Entonces tenemos un grafo conexo sin ciclos simples, es decir un árbol. Ahora estamos más cerca de la resolución.

Ahora como sabemos que es un árbol tenemos un único camino que conecta a cada nodo con el posible master. Y lo que buscamos es elegirlo de tal forma que se pueda llegar a todos los nodos en la menor cantidad de pasos posibles. El nodo master tiene que cumplir que su distancia máxima\footnote{es la distancia máxima desde el nodo hacia todos los demás mediante caminos simples} sea la mínima de entre todos los nodos.

Encontrar este nodo mediante la definición de arriba puede resultar muy costoso, ya que deberíamos calcular la máxima distancia a todos los nodos desde cada nodo, que nos conduce a una complejidad cuadrática.

En cambio vamos a resolverlo intentando encontrar un nodo que pertenezca a un máximo camino posible dentro del árbol, y además sea el centro de dicho camino, es decir la distancia desde él a ambos extremos no puede diferir en más de 1. Más adelante vamos a demostrar porque es correcto.

Para implementar el algoritmo utilizamos lo siguiente:
\begin{enumerate}
\item Una estructura para los nodos, que se compone de:


\begin{itemize}
\item Nodos\_Adyacentes: Una lista de adyacentes al nodo.
\item Prim\_Long\_Max y Seg\_Long\_Max: las longitudes de sus dos ramas más largas. 
\item Nodo\_Rama\_Max: el nodo hacia el que está volcada la rama más larga. 
\item Suma\_Longs: la suma de ambas longitudes.
\end{itemize}



\item Una lista de nodos(estructura del item anterior), llamada árbol.
\item Una función recursiva que parta de un nodo y vaya calculando los datos mencionados en el primer item, haciendo llamados a los adyacentes (salvo a aquel del que vino).
\item Finalmente con los datos almacenados en la lista de nodos, busco el nodo master.
\end{enumerate}

 Suponiendo que ya tenemos el arreglo de n nodos, cada uno en la posición que le corresponde según el número, con la información de sus adyacentes ya cargada, pasamos a ver el pseudocódigo del algoritmo recursivo y luego del que busca el master.

 \underline{*Nota*}: Dentro del algoritmo recursivo la función ActualizarMaximos actualiza los valores de las dos ramas más largas, y hacia donde está la rama más larga.
 Después del ciclo se usa una función llamada ActualizarArbol que actualiza los datos de la lista de nodos, para el nodo en el que se encuentra (dos ramas largas, suma de ramas y hacia donde va la rama más larga).
 
\begin{algorithm}
\caption{Algoritmo Recursivo}\label{Recursion}
 
\begin{algorithmic}[1]

\Function{CalcularRamaLarga}{$ nroNodo, nroPadre $} \Comment { El primero es el ID del nodo que quiero calcular y el segundo me dice de donde proviene}

\While{$QuedanHijos(Arbol[nroNodo])$}\Comment{cicla mientras todavía queden adyacentes por recorrer} 
	
	\State{$ hijo \gets ObtenerUnHijo(Arbol[nroNodo])$} \Comment { toma un nodo adyacente y lo guarda en Hijo}
	
	\If{$ hijo != NroPadre$}
		\State {$ rama = CalcularRamaLarga( hijo, nroNodo) $} \Comment {llamado recursivo}
		\State {$ ActualizarMaximos(rama, ramaMax1, ramaMax2, nodo\_rama\_Max1)$}\Comment {ver *Nota*}
	\EndIf
\EndWhile\

\State{$ ActualizarArbol (nroNodo, ramaMax1, ramaMax2, nodo\_rama\_Max1) $}\Comment {ver *Nota*}	
\State{$ return (ramaMax1 + 1) $} \Comment { esto devuelve al padre la longitud de la rama más larga hacia este nodo}

\EndFunction
\end{algorithmic}
\end{algorithm}

\begin{algorithm}
\caption{EncontrarMaster}
 
\begin{algorithmic}[1]
\Procedure{BuscarMaster}{}
\State{$nodo \gets Max\_Suma(Arbol)$} \Comment{recorre arbol, buscando el nodo con suma de ramas máxima}
\State{$ r1 = Arbol[nodo].ramaMax1$} \Comment {en r1 y r2 tengo las longitudes de las ramas más largas del nodo}
\State{$ r2 = Arbol[nodo].ramaMax2$}
	\While{r1 != r2 AND (r1 -1) != r2} \Comment{cicla mientras la diferencia entre r1 y r2 sea mayor a 1}
	\State{$ nodo = arbol[nodo].nodoRamaLarga$} \Comment{me muevo hacia el nodo con la rama más larga}
	\State{$ r1--$} \Comment{siempre me muevo hacia la rama de r1}
	\State{$ r2++$}
	\EndWhile
	\Comment{Cuando salgo del ciclo nodo contiene al master}
	
\EndProcedure
\end{algorithmic}
\end{algorithm}