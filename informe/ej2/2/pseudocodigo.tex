\subsection{Pseudocódigo}

Del enunciado del problema se pueden deducir ciertos datos que nos van a ayudar a dar con el algoritmo que buscamos.

Aunque resulte un poco obvio a esta altura, la forma de interpretar el problema es por medio de un grafo, en principio no orientado, donde los nodos representan servidores y las aristas los enlaces entre sí.

Sabemos por el enunciado que la entrada que recibimos, es la salida del ejercicio anterior, por lo tanto podemos afirmar:

\begin{itemize}
\item Desde un servidor (cualquiera) se debe poder llegar a todos los demás.
\item La cantidad de enlaces usados es mínima.
\end{itemize}

De lo primero se deduce que el grafo es conexo, no importa que par de nodos elija, existe un camino que los comunica.

El segundo punto nos permite afirmar que no hay circuitos simples (camino que salga de un nodo y vuelva al mismo, pasando a lo sumo una vez por cada arista). Si existiera uno, podría, fijando un master llegar a un cierto nodo n (distinto del master) que pertenece al circuito simple, de dos formas diferentes. (GRAFICAR O EXPLICAR MEJOR)

Entonces tenemos un grafo conexo sin ciclos simples, es decir un árbol. Ahora estamos m


Como vimos en la descripción, partimos de un nodo que es el master
