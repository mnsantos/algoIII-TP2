\subsection{Demostración}

Diremos que un conjunto de aristas factible es $prometedor$ si se puede extender para producir no sólo una solución, sino una solución óptima para nuestro problema. Si un conjunto de aristas prometedor ya es solución entonces esta solución debe ser óptima.

La demostración de la correctitud de algoritmo es por inducción sobre el número de aristas que hay en el conjunto T. Demostraremos que si T es prometedor en alguna fase del algoritmo, entonces sigue siendo prometedor al añandir una arista adicional. Cuando se detiene el agoritmo (T tiene C cantidad de aristas), T da una solución prometedora al problema y por lo tanto óptima.

Caso base: el conjunto vacío T es prometedor
Paso inductivo: supongamos que T es prometedor justo antes de que el algoritmo añada una nueva arista $e$ al conjunto T. T es un conjunto de aristas prometedor por hipótesis inductiva y $e$ es por definición una de las aristas más cortas que salen de B. Podemos asegurar que existe un $e$ que cumpla dichos requisitos ya que el enunciado asegura que todos los clientes pueden ser provistos por al menos una fábrica. Por lo tanto T $\cup$ $e$ también es prometedor.

Como el conjunto T es prometedor en todas las fases del algoritmo también lo será cuando finalice ya que ofrece una solción óptima del problema.
