\subsection{Implementación y Complejidad}

Para implementar Prim modificado utilizamos una cola de prioridad de manera tal que la elección de las aristas resultantes sea eficiente. A continuación un pseudocódigo de la implementación basado en el pseudocódigo presentado anteriormente pero de más bajo nivel.

\begin{algorithm}[H]
\caption{PrimModificado}\label{ej1}
\begin{algorithmic}[1]
\Procedure{PrimModificado}{$G=(V,E), int \ cantClientes,int \ cantFabricas$}
	\State T  $\shortleftarrow$ $\emptyset$
	\State Q  $\shortleftarrow$ $priority\_queue<arista>$
	\ForAll{$e=(u,v) \in E /$ $u$ es una fábrica y $v$ no} \Comment{recorro todas las fábricas y busco sus adyacentes}
		\State $push(e,Q)$
	\EndFor
	\For{$i \shortleftarrow 0$ to $cantClientes$}
		\State $e=(u,v)=obtenerMin(Q)$
		\State T $\cup$ $\{e\}$
		\ForAll{$j\in V\ / \ v$ es adyacente a j con $j\neq u$}
			\State $push((v,j),Q)$
		\EndFor
	\EndFor
	\State return T
\EndProcedure
\end{algorithmic}
\end{algorithm}

\begin{algorithm}[H]
\caption{obtenerMin}\label{ej1}
\begin{algorithmic}[1]
\Procedure{obtenerMin}{$G=(V,E), priority\_queue \ Q$}
	\State $bool$ $encontre=false$
	\State $res=(u,v)$
	\State $e'=(u',v')=Q.pop()$
	\While{$\neg encontre$}
	\If{$v'$ es una fábrica y $u'$ no tenía asociada una fábrica}
		\State $encontre=true$
		\State $res=e'$
	\Else
		\If{$v'$ es cliente y tanto $u'$ como $v'$ no tenían asociadas fábricas}
			\State $encontre=true$
			\State $res=e'$
		\EndIf
	\EndIf
	\EndWhile
	\State return $res$
\EndProcedure
\end{algorithmic}
\end{algorithm}

La complejidad de Prim Modificado dependerá de la estructura utilizada para ir eligiendo aristas. El cuerpo del for se ejecuta C veces (líneas 7-12). El obtener mínimo dentro del while se ejecutará O(R) veces en total ya que en los C llamados como mucho obtiene R aristas. Cada operación de push, pop cuesta dentro de obtenerMin cuesta log(R) lo que equivale a O(log $(F+C)^2$) ya que $(F+C)^2$ es la mayor cantidad de aristas posibles. Como F es menor que C (enunciado), log($(C+F)^2$) $\subset$ log($C^2$). Por lo tanto la complejidad equivale a O(2 log C) $\subset$ O(log C). El for que se encarga de pushear las aristas adyacentes a la cola de prioridad (líneas 10-12) también se ejecutará un total de R veces y la operación de push cuesta log(R) $\subset$ O(log C). Por lo tanto la complejidad total será de O(R*log(C)+R*log(C)) $\subset$ O(R*log(C)) como pedía el enunciado.
